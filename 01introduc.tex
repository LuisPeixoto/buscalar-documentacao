% Prof. Dr. Ausberto S. Castro Vera
% UENF - CCT - LCMAT - Curso de Ci\^{e}ncia da Computa\c{c}\~{a}o
% Campos, RJ,  2022
% Disciplina: Paradigma de Desenvolvimento Orientado a Objetos
% Aluno:


\chapterimage{sistemas.png} % Table of contents heading image
\chapter{ Introdu\c{c}\~{a}o}

\textit{Paradigma de Desenvolvimento de Sistemas Orientado a Objetos} \'{e} uma disciplina orientada a desenvolver um sistema utilizando  a metodologia orientada a Objetos en todas as etapas do Ciclo de Vida de Desenvolvimento de um Sistema (CVDS).  As refer\^{e}ncias bibliogr\'{a}ficas b\'{a}sicas a serem consultadas s\~{a}o: \cite{Dennis2014}, \cite{Engholm2013}, \cite{Guedes2011},  \cite{Sommerville2018} e \cite{Wazlawick2011}. Como bibliografia complementar ser\~{a}o considerados: \cite{Satzinger2012}, \cite{Shelly2012} e  \cite{Furgeri2013}.

Neste documento ser\~{a}o apresentadas as principais atividades realizadas para o desenvolvimento COMPLETO de uma aplica\c{c}\~{a}o OO.

O sistema a ser desenvolvido \'{e} .....


   \section{Escopo ou Contextualiza\c{c}\~{a}o do Sistema OO}

        \subsection{abcde}


        \subsection{defgh}

   \section{Objetivo do Sistema}
   Um objetivo Geral e 3 objetivos especificos

   \section{Justificativa}

   Justificar por que foi escolhido ou deve ser desenvolvido este sistema

